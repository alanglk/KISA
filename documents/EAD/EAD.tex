% #########################################################################
% #                       EXPLORACIÓN Y ANÁLISIS DE DATOS                 #
% #########################################################################

% Definición de nombres
\newcommand{\estudiante}{García Justel, Alan}
\newcommand{\titulo}{MÁSTER EN INGENIERÍA COMPUTACIONAL Y SISTEMAS INTELIGENTES}
\newcommand{\asignatura}{EXPLORACIÓN Y ANÁLISIS DE DATOS}
\newcommand{\portada}{common/no_signal.png}
\newcommand{\colorportada}{title_green}
\newcommand{\curso}{2024-2025}


% Notebook
\begin{document}
\newgeometry{bottom=2cm}

\begin{titlepage}
    % Logo de la universidad
    \begin{textblock*}{\textwidth}(2cm,0.4cm)
        \begin{center}
            \begin{minipage}{0.45\textwidth}
                \centering
                \includegraphics[width=\textwidth]{common/Logo_EHU.jpg}
            \end{minipage}\hfill
            \begin{minipage}{0.45\textwidth}
                \centering
                \includegraphics[width=0.8\textwidth]{common/Logo_KISA.png} 
            \end{minipage}
        \end{center}
    \end{textblock*}
    
    % Franja de color
    \begin{tikzpicture}[remember picture, overlay]
        \fill[\colorportada] (current page.north west) ++ (0,-3.01cm) rectangle (\paperwidth,-3cm);
    \end{tikzpicture}
    
    \begin{textblock*}{\paperwidth}(\dimexpr\parindent+\oddsidemargin+3em\relax,3.5cm)
        \begin{minipage}{\dimexpr\linewidth-7.5cm\relax}
            \color{white}
            \noindent\rule{\linewidth}{0cm}
            \textsf{ {\large \titulo}}
            \newline
            \newline \newline
            \textsf{\textbf{ {\Huge APUNTES DE ASIGNATURA }}}
        \end{minipage}
    \end{textblock*}
    
    % Nombre asignatura
    \vspace*{3.5cm}
    \begin{minipage}{\linewidth}
        \setlength{\baselineskip}{1.7\baselineskip}
        \centering
        \textsf{ \textbf{ {\LARGE \asignatura }}}
    \end{minipage}

    % Foto de portada
    \vspace*{0.5cm}
    \begin{figure}[H]
        \centering
        \includegraphics[width=10cm, height=8cm]{\portada}
    \end{figure}

    
    
    % Estudiante
    \vspace{0.2cm}
    \noindent {\footnotesize \textbf{Estudiante:} \estudiante}
    \newline
    \noindent\makebox[\linewidth]{\rule{\textwidth}{0.4pt}} % Línea horizontal

    % Curso y Fecha
    \vspace{0.1cm}    
    \noindent {\footnotesize \textbf{Curso: } \curso \hfill \textbf{Fecha:} \today }
\end{titlepage}

\restoregeometry
\setcounter{figure}{0} % Incluimos el título
\newpage
% Índices
% \tableofcontents\thispagestyle{empty} %\newpage
% \listoffigures\thispagestyle{empty}   %\newpage
% \listoftables\thispagestyle{empty}    %\newpage

\section{Visualización de Datos y Estadísticos}
se utiliza la notación nxp individuos por filas y atributos por columnas. Cuando nos refiramos al individuo siempre en forma de vector columna.

% Ejemplo
% Variables bianrias
% X=_ 'Tiene una anomalía'
% |0|
% |1|
% |0| media X = 1/5 = 0.2 -> nos devuelve la frecuencia del 1 (probabilidad de que ocurra 1)
% |0|
% |0|


Con una matriz de correlaciones solo podemos capturar o identificar relaciones lineales. Ejemplo de la correlación cuadrática.
\qs{}{is the set}
\sol{oooomagad}
\dfn{dos por tres}{Es un producto}
\ex{blabla}{gogo}
\end{document}