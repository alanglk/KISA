% #########################################################################
% #               METODOLOGÍAS Y TÉCNICAS DE INVESTIGACIÓN                #
% #########################################################################

% Definición de nombres
\newcommand{\estudiante}{García Justel, Alan}
\newcommand{\titulo}{MÁSTER EN INGENIERÍA COMPUTACIONAL Y SISTEMAS INTELIGENTES}
\newcommand{\asignatura}{METODOLOGÍAS Y TÉCNICAS DE INVESTIGACIÓN}
\newcommand{\portada}{common/no_signal.png}
\newcommand{\colorportada}{title_red}
\newcommand{\curso}{2024-2025}



% Notebook
\begin{document}

\newgeometry{bottom=2cm}

\begin{titlepage}
    % Logo de la universidad
    \begin{textblock*}{\textwidth}(2cm,0.4cm)
        \begin{center}
            \begin{minipage}{0.45\textwidth}
                \centering
                \includegraphics[width=\textwidth]{common/Logo_EHU.jpg}
            \end{minipage}\hfill
            \begin{minipage}{0.45\textwidth}
                \centering
                \includegraphics[width=0.8\textwidth]{common/Logo_KISA.png} 
            \end{minipage}
        \end{center}
    \end{textblock*}
    
    % Franja de color
    \begin{tikzpicture}[remember picture, overlay]
        \fill[\colorportada] (current page.north west) ++ (0,-3.01cm) rectangle (\paperwidth,-3cm);
    \end{tikzpicture}
    
    \begin{textblock*}{\paperwidth}(\dimexpr\parindent+\oddsidemargin+3em\relax,3.5cm)
        \begin{minipage}{\dimexpr\linewidth-7.5cm\relax}
            \color{white}
            \noindent\rule{\linewidth}{0cm}
            \textsf{ {\large \titulo}}
            \newline
            \newline \newline
            \textsf{\textbf{ {\Huge APUNTES DE ASIGNATURA }}}
        \end{minipage}
    \end{textblock*}
    
    % Nombre asignatura
    \vspace*{3.5cm}
    \begin{minipage}{\linewidth}
        \setlength{\baselineskip}{1.7\baselineskip}
        \centering
        \textsf{ \textbf{ {\LARGE \asignatura }}}
    \end{minipage}

    % Foto de portada
    \vspace*{0.5cm}
    \begin{figure}[H]
        \centering
        \includegraphics[width=10cm, height=8cm]{\portada}
    \end{figure}

    
    
    % Estudiante
    \vspace{0.2cm}
    \noindent {\footnotesize \textbf{Estudiante:} \estudiante}
    \newline
    \noindent\makebox[\linewidth]{\rule{\textwidth}{0.4pt}} % Línea horizontal

    % Curso y Fecha
    \vspace{0.1cm}    
    \noindent {\footnotesize \textbf{Curso: } \curso \hfill \textbf{Fecha:} \today }
\end{titlepage}

\restoregeometry
\setcounter{figure}{0} % Incluimos el título
\newpage

% Índices
% \tableofcontents\thispagestyle{empty} %\newpage
% \listoffigures\thispagestyle{empty}   %\newpage
% \listoftables\thispagestyle{empty}    %\newpage

\section{Tema 1: ¿Qué es la investigación?}
Existen varios tipos de documentos científicos. Por un lado tenemos las tesis de doctorado o de máster en las que se sintetiza un trabajo de investigación que aporte nuevos resultados sobre el presente estado del arte del momento. Por otro lado, existen los artículos científicos que se presentan en revistas y congresos. Estos últimos aportan mucho <<Networking>> de tal forma que te permite saber las líneas de investigación existentes sobre un tema actual.

Finalmente, para poder investigar es necesario contar con una <<Solicitud de Proyecto de Investigación>>. Es un documento científico importante en el que además hay que hacer un análisis exahustivo del estado del arte y es necesario para conseguir fondos para el proyecto. 

\subsection{¿Cómo escribir una tésis?}
El primer paso es la elección y justificación del tema o, si ya se tiene elegido, búsquedas de documentos relacionados con el tema a tratar. Para ello, existen varios portales como Google Scholar \url{https://scholar.google.es/}, TESEO \url{https://www.educacion.gob.es/teseo/irGestionarConsulta.do}, ProQuest \url{https://www.proquest.com/}... Además, hay que justificar el problema y mostrar la utilidad del trabajo a realizar (parte del contexto) y un análisis del estado del arte: antecedentes, métodos actuales, limitaciones y condicionantes etc. Otro aspecto muy importante es el planteamiento de los objetivos de la tésis.

\textbf{TODO: COMPLETAR ESTA LISTA PARA EL TFM}

Estructura de una tésis:
1. Introducción
2. Contexto / Comparativa con el estado del arte
3. 



Papers interesantes para el TFM:
https://ieeexplore.ieee.org/abstract/document/8206396
https://link.springer.com/chapter/10.1007/978-3-031-19769-7_23
https://link.springer.com/article/10.1007/s00138-023-01400-7?fromPaywallRec=false
https://link.springer.com/chapter/10.1007/978-981-99-8435-0_29?fromPaywallRec=false




\end{document}