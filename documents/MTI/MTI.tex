% #########################################################################
% #               METODOLOGÍAS Y TÉCNICAS DE INVESTIGACIÓN                #
% #########################################################################

% Definición de nombres
\newcommand{\estudiante}{García Justel, Alan}
\newcommand{\titulo}{MÁSTER EN INGENIERÍA COMPUTACIONAL Y SISTEMAS INTELIGENTES}
\newcommand{\asignatura}{METODOLOGÍAS Y TÉCNICAS DE INVESTIGACIÓN}
\newcommand{\portada}{common/no_signal.png}
\newcommand{\colorportada}{title_red}
\newcommand{\curso}{2024-2025}



% Notebook
\begin{document}

\newgeometry{bottom=2cm}

\begin{titlepage}
    % Logo de la universidad
    \begin{textblock*}{\textwidth}(2cm,0.4cm)
        \begin{center}
            \begin{minipage}{0.45\textwidth}
                \centering
                \includegraphics[width=\textwidth]{common/Logo_EHU.jpg}
            \end{minipage}\hfill
            \begin{minipage}{0.45\textwidth}
                \centering
                \includegraphics[width=0.8\textwidth]{common/Logo_KISA.png} 
            \end{minipage}
        \end{center}
    \end{textblock*}
    
    % Franja de color
    \begin{tikzpicture}[remember picture, overlay]
        \fill[\colorportada] (current page.north west) ++ (0,-3.01cm) rectangle (\paperwidth,-3cm);
    \end{tikzpicture}
    
    \begin{textblock*}{\paperwidth}(\dimexpr\parindent+\oddsidemargin+3em\relax,3.5cm)
        \begin{minipage}{\dimexpr\linewidth-7.5cm\relax}
            \color{white}
            \noindent\rule{\linewidth}{0cm}
            \textsf{ {\large \titulo}}
            \newline
            \newline \newline
            \textsf{\textbf{ {\Huge APUNTES DE ASIGNATURA }}}
        \end{minipage}
    \end{textblock*}
    
    % Nombre asignatura
    \vspace*{3.5cm}
    \begin{minipage}{\linewidth}
        \setlength{\baselineskip}{1.7\baselineskip}
        \centering
        \textsf{ \textbf{ {\LARGE \asignatura }}}
    \end{minipage}

    % Foto de portada
    \vspace*{0.5cm}
    \begin{figure}[H]
        \centering
        \includegraphics[width=10cm, height=8cm]{\portada}
    \end{figure}

    
    
    % Estudiante
    \vspace{0.2cm}
    \noindent {\footnotesize \textbf{Estudiante:} \estudiante}
    \newline
    \noindent\makebox[\linewidth]{\rule{\textwidth}{0.4pt}} % Línea horizontal

    % Curso y Fecha
    \vspace{0.1cm}    
    \noindent {\footnotesize \textbf{Curso: } \curso \hfill \textbf{Fecha:} \today }
\end{titlepage}

\restoregeometry
\setcounter{figure}{0} % Incluimos el título
\newpage

% Índices
% \tableofcontents\thispagestyle{empty} %\newpage
% \listoffigures\thispagestyle{empty}   %\newpage
% \listoftables\thispagestyle{empty}    %\newpage

\section{Introducción al mundo de la investigación}
Existen varios tipos de documentos científicos. Por un lado tenemos las tesis de doctorado o de máster en las que se sintetiza un trabajo de investigación que aporta nuevos resultados sobre el presente estado del arte del momento. Por otro lado, existen los artículos científicos que se presentan en revistas y congresos. Estos últimos aportan mucho <<Networking>> de tal forma que te permite saber las líneas de investigación existentes sobre un tema actual.

Finalmente, para poder investigar es necesario contar con una <<Solicitud de Proyecto de Investigación>>. Es un documento científico importante en el que además hay que hacer un análisis exahustivo del estado del arte y es necesario para conseguir fondos para el proyecto. 

\subsection{Empezar a escribir una tésis}
El primer paso es la elección y justificación del tema o, si ya se tiene elegido, búsquedas de documentos relacionados con el tema a tratar. Para ello, existen varios portales como \href{https://scholar.google.es/}{Google Scholar}, \href{https://www.educacion.gob.es/teseo/irGestionarConsulta.do}{TESEO}, \href{https://www.proquest.com/}{ProQuest}... Además, hay que justificar el problema y mostrar la utilidad del trabajo a realizar (parte del contexto) y un análisis del estado del arte: antecedentes, métodos actuales, limitaciones y condicionantes etc. Otro aspecto muy importante es el planteamiento de los objetivos de la tésis.

Antes de comenzar con la escritura, es muy recomendable desglosar el tema y dividirlo en capítulos. En resúmen, es recomendable realizar una exploración sobre las líneas actuales de investigación afines al tema a tratar y crear el índice del documento antes de comenzar con la escritura.

\ex{Estructura de una tésis}{
    \begin{enumerate}
        \item Introducción
        \item Estado del arte
        \item Cuerpo principal de la tésis
        \item Conclusiones
        \item Desarrollos futuros
        \item Referencias
        \item Apéndices
    \end{enumerate}
}

\subsection{Revisión del estado del arte}
La revisión del estado del arte es un proceso metódico para identificar, evaluar e interpretar el trabajo hecho por otros investigadores sobre un tema de interés. No es muy común el referenciar tésis doctorales o de máster durante esta revisión y, aunque no habría problema en referenciar un par de ellas, sí que es muy importante referenciar artículos de primer nivel para dar peso al análisis del estado del arte.

En definitiva, la sección de la revisión del estado del arte es el resultado de un proceso de análisis, síntesis y comprensión que se refleja a modo de una pequeña <<historia>>.

\qs{Búsqueda de grupos de investigación}{
Se tienen que buscar grupos de investigación de departamentos de informática actuales en la universidad de Granada (UGr) y recoger cuáles son sus líneas de investigación principales.
}

\sol{
    Los siguientes grupos de investigación se han obtenido del siguiente \href{https://lsi.ugr.es/investigacion/grupos#title0}{link} a fecha de $17/09/2024$:
    {\tiny
    \begin{enumerate}[itemsep=0mm]
        \item Grupo de Especificación, Desarrollo y Evolución de Software (GEDES)\begin{itemize}[nosep]
            \item Software para discapacidad
            \item Enseñanza Virtual
            \item Informática y accesibilidad
            \item Modelos epistemológicos
        \end{itemize}
        \item Informática Gráfica y Realidad Virtual
        \begin{itemize}[nosep]
            \item Realidad Virtual.
            \item Digitalización 3D de patrimonio histórico
            \item Representación y visualización de modelos 3D para aplicaciones médicas
            \item Animación por ordenador y visualización expresiva.
            \item Ingeniería de aplicaciones gráficas.
            \item Visualización de grandes modelos.
            \item Maquetación Virtual.
        \end{itemize}
        \item Modelling and Development of Advanced Software Systems (MYDASS)
        \begin{itemize}[nosep]
            \item Human Computer Interaction
            \item Assistive technologies
            \item Frameworks and platforms for mobile systems
            \item Sensor Networks and domotics for Health Systems
        \end{itemize}
        \item Sistemas Concurrentes
        \begin{itemize}[nosep]
            \item Verificación de software asistida por computador
            \item Métodos formales en sistemas concurrentes
            \item Computación de tiempo real y empotrada para sistemas ubicuos y de inteligencia ambiental
            \item Middlewares y marcos de desarrollo para sistemas empotrados de tiempo real distribuidos
            \item Sistemas de medición distribuido y de instrumentación 
        \end{itemize} 
        \item Sistemas de Diálogo Hablado y Multimodal
        \begin{itemize}[nosep]
            \item Speech recognition, understanding and generation
            \item Dialogue management
            \item User simulation
            \item Affective computing
            \item Multimodal interaction
            \item Ambient Intelligence (AmI)
            \item Generation of computer personality models
        \end{itemize}
        \item Tecnologías y Aplicaciones de Realidad Virtual, Interacción y Simulación (TARVIS)
        \begin{itemize}[nosep]
            \item 3D Scanning and dissemination of Cultural Heritage
            \item Numeric simulation in high performance graphic devices.
            \item New technics on Human-Computer Interaction
            \item Solid and Volume Modelling
            \item Medical Applications
        \end{itemize}
    \end{enumerate}
    }

    Buscar también en \href{https://www.aepia.org/grupos-de-investigacion/}{esta} web más actualizada.
}

Otra herramienta interesante para realizar búsquedas de artículos científicos es el portal \href{https://www.sciencedirect.com/}{ScienceDirect} en el que se recogen numerosas revistas. Para obtener datos sobre el número de referencias, cuartiles y otras estadísticas a cerca de una revista, el portal \href{https://jcr.clarivate.com/jcr/home?app=jcr&Init=Yes&authCode=null&SrcApp=IC2LS}{Journal Citation Reports} (JCR) viene muy bien. También puede servir \href{https://www.webofscience.com/wos/alldb/basic-search}{WebOfScience}.

Existen varias formas de presentar o depositar una tésis doctoral:
\begin{itemize}
    \item Clásica
    \item Tésis por compilación de artículos: 3 Artículos publicados; JCR mínimo una en primero o segundo cuartil
\end{itemize}

\nt{Artículos interesantes para el TFM}{
    \href{https://www.sciencedirect.com/journal/robotics-and-autonomous-systems}{Robotics and Autonomous Systems}
    \begin{itemize}
        \url{https://www.sciencedirect.com/science/article/pii/S0921889023001306}
        \url{https://www.sciencedirect.com/science/article/pii/S0921889008000341}
        \url{https://www.sciencedirect.com/science/article/pii/S0921889014002851}
    \end{itemize}

    \href{https://www.sciencedirect.com/journal/image-and-vision-computing}{Image and Vision Computing}
    \begin{itemize}
        \url{https://www.sciencedirect.com/science/article/pii/S0262885624002543}
    \end{itemize}
}

\nt{Otros links interesantes}{
State of the art of semantic segmentation for autonomous driving:
    \begin{enumerate}
        \item \url{https://ieeexplore.ieee.org/abstract/document/8206396}
        \item \url{https://link.springer.com/chapter/10.1007/978-3-031-19769-7_23}
        \item \url{https://link.springer.com/article/10.1007/s00138-023-01400-7?fromPaywallRec=false}
        \item \url{https://link.springer.com/chapter/10.1007/978-981-99-8435-0_29?fromPaywallRec=false}
    \end{enumerate}
}

\qs{Ejercicio Práctico}{
    Realiza una revisión bibliográfica sobre <<uso de inteligencia artificial para reconocimiento de habla>> y expçortala a tu cuenta de RefWorks.
    Sigue estos pasos:
    \begin{enumerate}
        \item 
    \end{enumerate}

}

Bases de datos:
- Web Of Science
- Scopus
- Inspec
- Biblioteca UPV/EHU  

Bases de datos referenciales vs. Bibliotecas digitales
Las bases de datos tienen todas las referencias a todos los textos mientras que las bibliotecas solo tienen las publicaciones de una editorial en particular.
Las bibliotecas tienen los textos completos mientras que las bases de datos contienen las referencias a dichas paginas de la biblioteca.

En conclusión, a la hora de hacer búsquedas directas, las bases de datos referenciales deben ser la primera opción de búsqueda.






\end{document}