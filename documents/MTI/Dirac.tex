\title{Paul Adrien Maurice Dirac}
\author{Alan García}

\begin{document}
\maketitle \newpage

\section{Biografía}
Paul Adrien Maurice Dirac fué un físico teórico que contribuyó notablemente al desarrollo de la mecánica y la electrodinámica cuánticas y que, debido a sus contribuciones, se ha convertido en uno de los científicos más influyentes del siglo XX. 

Paul Dirac nació el 8 de agosto de 1902 en Brístol, Reino Unido. Sus padres eran Charles Adrien, que ejercía de profesor de francés en la \textit{Bishop Primary School}, y Florence Hannah, originaria de Cornualles con padres marineros. Además, tenía una hermana pequeña llamada Beatrice Isabelle y un hermano mayor, Reginald Charles Felix, que desgraciadamente se suicidó en 1924. Dirac describió su infancia como infeliz debido al autoritarismo y control que ejercía su padre, que se extendía incluso en el colegio, y que motivaron su carácter reservado \ref{RefWorks:farmelo2009strangest}.

Después de completar sus estudios primarios accedió a la \textit{Merchant Venturer’s Secondary School}, una institución de la universidad de Bristol que enfatizaba las ciencias y tecnologías, para luego ingresar en la Universidad de Bristol graduándose en ingeniería eléctrica en 1921. Sin embargo, después un breve periodo trabajando decidió estudiar matemáticas en la misma universidad consiguiendo el título dos años más tarde. Finalmente, fue admitido como investigador en la universidad de Cambridge y recibió el doctorado en física en 1926 por su trabajo en mecánica cuántica. 

Además, en 1932 fue reconocido con el título de \textit{Lucasian Professor of Mathematics} por la universidad de Cambridge en la que ejerció de profesor hasta 1969. En 1933 compartió el premio novel de física junto con Erwin Schrödinger por <<El descubrimiento de nuevas formas de la teoría atómica>>. En 1937 se casó con Margit Wigner y, durante los últimos años de su vida, siguó investigando en la Universidad Estatal de Florida en Tallahassee hasta su fallecimiento en 1984.

\section{Investigación científica}
Todavía en la segunda década del siglo XX la mecánica cuántica era un campo incipiente en el que los resultados experimentales no se correspondían con los resultados esperados por las leyes de la mecánica clásica aplicadas. 

En este contexto, se encontraban numerosos científicos trabajando sobre los fenómenos atómicos como Niels Bohr, quien publicó su propio modelo atómico \ref{RefWorks:bohr1913i} en 1913 introduciendo la teoría de las órbitas cuantificadas. Además, en 1925 Werner Heisenberg ideó la mecánica matricial \ref{RefWorks:heisenberg1925quantum-theoretical}, una nueva formulación de la mecánica cuántica en la que se sugiere que no son las ecuaciones de la mecánica clásica las que tienen algún defecto, sino que son las operaciones matemáticas mediante las cuales se deducen los resultados físicos las que requieren modificación.

Paul Dirac comenzó su carrera científica en 1925 publicando varios artículos como \textit{The Fundamental Equations Of Quantum Mechanics} \ref{RefWorks:dirac1925fundamental} que tomaban de base el trabajo de Bohr y unificaban las teorías de Heisenberg y Schrödinger en un único modelo para describir el estado de un sistema atómico utilizando una nueva notación conocida como <<braket Dirac>>.

En 1928, propuso la \textit{ecuación Dirac} como una ecuación relativista que describe al electrón \ref{RefWorks:dirac1928quantum}. Este trabajo permitió a Dirac predecir la existencia del positrón, el cual fue observado por primera vez por Carl Anderson en 1932, y contribuyó también a explicar el spin como un fenómeno relativista. Más tarde en 1930 Dirac publicó \textit{The Principles of Quantum Mechanics} \ref{RefWorks:dirac1958principles}, un libro que recogía toda su investigación y que se ha posicionado como uno de los libros más referentes de mecánica cuántica.

Paul Dirac propuso en 1931 la existencia de un único monopolo magnético sería suficiente para explicar la cuantización de la carga eléctrica. Sin embargo, todavía no se ha descubierto experimentalmente este monopolo \ref{RefWorks:dirac1931quantised}.


\printbibliography
\end{document}