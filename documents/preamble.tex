% #########################################################################
% #                           PACKAGE IMPORTS                             #
% #########################################################################

\usepackage[spanish]{babel}             % Español para los nombres
\usepackage[utf8]{inputenc}             % Codificación de entrada
\usepackage[T1]{fontenc}                % Codificación de fuente (escribir castellano)
\usepackage{lmodern}                    % Fuente (la default no es compatible con castellano)
\usepackage{csquotes}                   % Para comillas y otras anotaciones del castellano
\usepackage{enumitem}					% Mejora los enumerates
\usepackage{fancyhdr}                   % Encabezado y Pie de pagina
\usepackage{tocloft}                    % Puntos en los índices

\usepackage{url}                        % 
\usepackage{hyperref}                   % Links y referencias dentro del texto
\usepackage{nameref}                    % Para poder referenciar por nombre
\usepackage{acronym}                    % Para incluir las listas de abreviaturas
\usepackage{multicol}                   % Permite crear espacios con columnas verticales
\usepackage{booktabs}                   % Para mejores líneas horizontales en tablas
\usepackage{parskip}                    % Eliminar sangrado y añadir espacio entre párrafos
\usepackage{titlesec}                   % Crear subsubsub secciones
\usepackage{amsmath}                    % Funciones para ecuaciones
\usepackage{amssymb}					% Funciones para ecuaciones
\usepackage{blkarray}                   % Matrices con anotaciones
\usepackage{algorithm}                  % Algoritmos con formalismos
\usepackage{algpseudocode}              % También para algoritmos
\usepackage{listings}                   % Para insertar codigo

\usepackage{array}                      % Para personalizar columnas de tabla
\usepackage{tabularx}                   % Para tablas de ancho variable

\usepackage{graphicx}                   % Para incluir imágenes
\usepackage[most]{tcolorbox}            % Cuadros para ejercicios etc
\usepackage{varwidth}                   % Para los cuadros tambien
\usepackage{subcaption}                 % Subfiguras
\usepackage[table]{xcolor}              % Definir y utilizar colores
\usepackage{tikz}                       % Dibujar formas, figuras, rectas, intersecciones...
\usepackage[absolute,overlay]{textpos}  % Posición absoluta para textos
\usepackage{geometry}                   % Para controlar los márgenes

\usepackage{float}

\usepackage[backend=biber, sorting=none]{biblatex} % Referencias (bibliografía / webgrafía)


% #########################################################################
% #                              COLOR SETUP                              #
% #########################################################################

% Definición de colores
\definecolor{myg}{RGB}{56, 140, 70}
\definecolor{myb}{RGB}{45, 111, 177}
\definecolor{myexamplebg}{HTML}{F2FBF8}
\definecolor{myexamplefr}{HTML}{88D6D1}
\definecolor{myexampleti}{HTML}{2A7F7F}
\definecolor{mynotebg}{HTML}{fff9e7}
\definecolor{mynotefr}{HTML}{ffcd33}
\definecolor{mynoteti}{HTML}{e6ad00}

\definecolor{ehu_blue}{HTML}{376092}
\definecolor{title_red}{HTML}{f34157}
\definecolor{title_green}{HTML}{009a5e}
\definecolor{title_blue}{HTML}{034d9a}
\definecolor{title_orange}{HTML}{ffa800}
\definecolor{link_color}{HTML}{36AEB4}
\definecolor{reference_color}{HTML}{0F3133}
\definecolor{cite_color}{HTML}{ff7f42}
\definecolor{table_gray}{HTML}{C0C0C0}
\definecolor{table_red}{HTML}{fb4952}

% #########################################################################
% #                        PACKAGE CONFIGURATION                          #
% #########################################################################
% Nombre de las tablas y de los indices a español
\renewcommand{\cftsecleader}{\cftdotfill{\cftdotsep}} 
\renewcommand{\spanishtablename}{Tabla}
\renewcommand{\spanishlisttablename}{Índice de tablas}

% Encabezado y Pie de pagina
\pagestyle{fancy}                       % Estilo de las páginas
\fancyhf{}
\fancyhead[L]{KISA}
\fancyhead[R]{\leftmark}
\fancyfoot[L]{UPV/EHU}
\fancyfoot[R]{\thepage}
\renewcommand{\footrulewidth}{0.4pt} % Línea horizontal en el pie de página
\setlength{\headheight}{15.5pt}

% Configuración de hyperref para diferentes tipos de enlaces
\hypersetup{
    colorlinks=true,
    linkcolor=reference_color,
    citecolor=cite_color,
    filecolor=link_color,
    urlcolor=link_color,
}

% Estilos de listings
% Configuración del estilo para resaltar el código de ROS
\lstdefinestyle{ros}{
    backgroundcolor=\color{white}, % Cambiar el color de fondo
    basicstyle=\small\ttfamily, % Texto más pequeño
    breaklines=true,
    language=C,
    morekeywords={int32, float64},
    keywordstyle=\color{blue},
    commentstyle=\color{green!40!black},
    stringstyle=\color{red},
    breakatwhitespace=false,         
    breaklines=true,                 
    captionpos=b,                    
    keepspaces=true,                 
    numbers=left,                    
    numbersep=5pt,                  
    showspaces=false,                
    showstringspaces=false,
    showtabs=false,                  
    tabsize=2
}

\lstdefinestyle{R}{
    backgroundcolor=\color{white}, % Cambiar el color de fondo
    basicstyle=\small\ttfamily, % Texto más pequeño
    breaklines=true,
    language=R,
    morekeywords={int32, float64},
    keywordstyle=\color{blue},
    commentstyle=\color{green!40!black},
    stringstyle=\color{red},
    breakatwhitespace=false,         
    breaklines=true,                 
    captionpos=b,                    
    keepspaces=true,                 
    numbers=left,                    
    numbersep=5pt,                  
    showspaces=false,                
    showstringspaces=false,
    showtabs=false,                  
    tabsize=2
}

% Notas en Figuras
\newcommand\fnote[1]{\captionsetup{width=0.8\linewidth, font=footnotesize}\caption*{#1}}

% Crear subsubsub secciones
\titleclass{\subsubsubsection}{straight}[\subsection]
\newcounter{subsubsubsection}[subsubsection]
\renewcommand\thesubsubsubsection{\thesubsubsection.\arabic{subsubsubsection}}
\renewcommand\theparagraph{\thesubsubsubsection.\arabic{paragraph}} % optional; useful if paragraphs are to be numbered
\titleformat{\subsubsubsection}
  {\normalfont\normalsize\bfseries}{\thesubsubsubsection}{1em}{}
\titlespacing*{\subsubsubsection}
{0pt}{3.25ex plus 1ex minus .2ex}{1.5ex plus .2ex}
\makeatletter
\renewcommand\paragraph{\@startsection{paragraph}{5}{\z@}%
  {3.25ex \@plus1ex \@minus.2ex}%
  {-1em}%
  {\normalfont\normalsize\bfseries}}
\renewcommand\subparagraph{\@startsection{subparagraph}{6}{\parindent}%
  {3.25ex \@plus1ex \@minus .2ex}%
  {-1em}%
  {\normalfont\normalsize\bfseries}}
\def\toclevel@subsubsubsection{4}
\def\toclevel@paragraph{5}
\def\toclevel@paragraph{6}
\def\l@subsubsubsection{\@dottedtocline{4}{7em}{4em}}
\def\l@paragraph{\@dottedtocline{5}{10em}{5em}}
\def\l@subparagraph{\@dottedtocline{6}{14em}{6em}}
\makeatother
\setcounter{secnumdepth}{4}
\setcounter{tocdepth}{4}

% #########################################################################
% #                              MATH BOXES                               #
% #########################################################################

% Ejercicio
\makeatletter
\newtcbtheorem{question}{Question}{enhanced,
	breakable,
	colback=white,
	colframe=myb!80!black,
	attach boxed title to top left={yshift*=-\tcboxedtitleheight},
	fonttitle=\bfseries,
	title={#2},
	boxed title size=title,
	boxed title style={%
			sharp corners,
			rounded corners=northwest,
			colback=tcbcolframe,
			boxrule=0pt,
		},
	underlay boxed title={%
			\path[fill=tcbcolframe] (title.south west)--(title.south east)
			to[out=0, in=180] ([xshift=5mm]title.east)--
			(title.center-|frame.east)
			[rounded corners=\kvtcb@arc] |-
			(frame.north) -| cycle;
		},
	#1
}{def}
\makeatother

% Ejemplo
\makeatletter
\newtcbtheorem[number within=section]{example}{Example}
{%
	colback = myexamplebg
	,breakable
	,colframe = myexamplefr
	,coltitle = myexampleti
	,boxrule = 1pt
	,sharp corners
	,detach title
	,before upper=\tcbtitle\par\smallskip
	,fonttitle = \bfseries
	,description font = \mdseries
	,separator sign none
	,description delimiters parenthesis
}{ex}
\makeatother

% Nota
\makeatletter
\newtcbtheorem[number within=section]{note}{Note}
{%
	colback = mynotebg
	,breakable
	,colframe = mynotefr
	,coltitle = mynoteti
	,boxrule = 1pt
	,sharp corners
	,detach title
	,before upper=\tcbtitle\par\smallskip
	,fonttitle = \bfseries
	,description font = \mdseries
	,separator sign none
	,description delimiters parenthesis
}{nt}
\makeatother

% Definicion
\makeatletter
\newtcbtheorem[number within=section]{definition}{Definition}{enhanced,
	before skip=2mm,after skip=2mm, colback=red!5,colframe=red!80!black,boxrule=0.5mm,
	attach boxed title to top left={xshift=1cm,yshift*=1mm-\tcboxedtitleheight}, varwidth boxed title*=-3cm,
	boxed title style={frame code={
					\path[fill=tcbcolback]
					([yshift=-1mm,xshift=-1mm]frame.north west)
					arc[start angle=0,end angle=180,radius=1mm]
					([yshift=-1mm,xshift=1mm]frame.north east)
					arc[start angle=180,end angle=0,radius=1mm];
					\path[left color=tcbcolback!60!black,right color=tcbcolback!60!black,
						middle color=tcbcolback!80!black]
					([xshift=-2mm]frame.north west) -- ([xshift=2mm]frame.north east)
					[rounded corners=1mm]-- ([xshift=1mm,yshift=-1mm]frame.north east)
					-- (frame.south east) -- (frame.south west)
					-- ([xshift=-1mm,yshift=-1mm]frame.north west)
					[sharp corners]-- cycle;
				},interior engine=empty,
		},
	fonttitle=\bfseries,
	title={#2},#1}{def}
\makeatother